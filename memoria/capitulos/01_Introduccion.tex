\chapter{Introducción}
Hoy en día debido a los avances tecnológicos, a la digitalización de la información y al uso intensivo de redes sociales se está generando inmensas cantidades de datos inabarcables por el ser humano y que solo pueden ser tratados por las técnicas de machine learning.  

Dentro de este campo los algoritmos que más éxito tienen son los conocidos como \textbf{redes neuronales}. Su propósito es el de procesar la información, ``aprender'' de ella y ofrecer un resultado que dependerá de su diseño, pudiendo ser una predicción numérica (problemas de regresión), una clasificación (problemas de clasificación) o una combinación de los anteriores. Para que estas redes puedan resolver problemas más complejos necesitan un incremento en la potencia computacional que les permita analizar la mayor cantidad de datos posibles. Además, debido a la naturaleza de ciertos problemas se necesitan que estas redes se puedan ejecutar en tiempo real, como es el caso de la conducción autónoma de vehículos. 

Estas necesidades son cada vez más difíciles de conseguir debido principalmente a varios problemas como son la ralentización de la ley de Moore, la separación entre memoria y procesador debido a la arquitectura de Von Neuman, y el alto consumo energético de los procesos de entrenamiento los cuales son muy contaminantes.  

Debido a todos estos problemas un campo que está emergiendo y es cada día más importante es el de la \textbf{computación neuromórfica}. 


\newpage
Aquí comienza bien